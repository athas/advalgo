\documentclass[a4paper, oneside, final]{memoir}
\usepackage{amsmath}
\begin{document}
\title{Max-Flow/Linear programming}
\author{Troels\\Ronni\\Jakob}
\maketitle

\setcounter{chapter}{1}

\section{\textit{Show how this problem can be solved using max-flow
    and briefly argue that your method works}}

(We interpret \textit{do not share a single edge} as \textit{do not
  share one or more edges}.)

The problem can be solved by assigning capacity $1$ to each edge in
the planar graph, then using any max-flow algorithm.  In model terms,
each edge can thus only be part of a single unique bike path, each
path will carry a flow of size $1$, and the maximum flow of the entire
network will thus be equal to the number of of different paths.

\section{\textit{Could he still solve the problem using max-flow, and
    if so how?}}

Vertex capacities can be assigned by splitting each vertex $v$ into
two vertices $v_1, v_2$ and joining them by an edge $e$ of the
capacity we wish the vertex to have, thus reducing the problem to the
earlier one of vertex disjointness.  By assigning each vertex in the
graph the capacity $1$, we require that all paths through the flow are
vertex-disjoint.

\section{\textit{Assuming there are n vertices and m edges in the
    Copenhagen graph, how fast could Carl solve this problem
    (asymptotic running-time)?}}

The Ford-Fulkerson algorithm has a worst-case asymptotic running time
of $O(E|f^\star|)$.  As all vertices and edges have capacity $1$,
$|f^\star|$ is bounded above by $E$.  Hence, the run-time complexity
is $O(m^2)$.  However, $|f^\star|$ is also bounded above by $\kappa$,
the vertex connectivity of the graph, and the complexity can thus be
reduced further to $O(m\kappa)$.  $\kappa$ is $O(n)$, so the
complexity will be $O(n m)$.  In practice, $\kappa$ will usually be
much smaller than $n$.

In comparison, the Edmunds-Karp algorithm runs in $O(V E^2)$, that is
$O(n m^2)$ in our case.

\section{\textit{Do you think he could solve it faster using some
    other method than max-flow?}}

By breadth-first iteration we can find all disjoint paths by removing
an edge from the graph whenever we move along it.

\section{\textit{Formulate this as a linear program and give a brief
    description of how you arrived at your solution.}}

\def\Sum{\displaystyle\sum}

From the textbook, we recalled the following linear program for
solving the \textit{minimum-cost flow} problem:

\begin{tabular}{lrcll}
  Minimize & $\Sum_{(u,v)\in E}a(u,v)f_{u v}$ &&& \\
  subject to & $f_{u v}$ & $\leq$ & $c(u,v)$ & for each $u,v \in
  V$ \\
  & $\Sum_{v\in V}f_{v u} - \Sum_{v\in V}f_{u v}$ &$=$& $0$ & for
  each $u\in V-{\{s,t\}}$ \\
  & $\Sum_{v\in V}f_{s v}-\Sum_{v\in V}f_{v s}$ &$=$& $d$ \\
  & $f_{u v}$ &$\geq$& $0$ & for each $u, v \in V$ \\
\end{tabular}

We essentially need to solve a \textit{maximum-cost} problem in which
$c(u,v)=1$ for all $(u,v)\in E$.  A naive attempt would simply change
``minimize'' to ``maximize'', but as our graph is not directed, this
would lead to paths that backtrack in order to double their interest.
We assume this is not intendedn and thus we change the capacity
constraint to $f_{u v}+f_{v u} \leq 1$.  This enforces that a path can
only be traversed once in any direction.  We also require that all
flows are either $0$ or $1$.  Our final linear program is thus:

\begin{tabular}{lrcll}
  Maximize & $\Sum_{(u,v)\in E}a(u,v)f_{u v}$ &&& \\
  subject to & $f_{u v}+f_{v u}$ & $\leq$ & $1$ & for each $u,v \in
  V$ \\
  & $\Sum_{v\in V}f_{v u} - \Sum_{v\in V}f_{u v}$ &$=$& $0$ & for
  each $u\in V-{\{s,t\}}$ \\
  & $\Sum_{v\in V}f_{s v}-\Sum_{v\in V}f_{v s}$ &$=$& $d$ \\
  & $f_{u v}$ &$\geq$& $0$ & for each $u, v \in V$ \\
\end{tabular}

\section{Solve max-edge-independent-paths to optimality using CPLEX}

The max flow through the network is $4$, hence that is also the number
of edge-independent paths (though there are only $3$
vertex-independent paths).

\section{Solve d-max-interest-paths to optimality using CPLEX}

\begin{description}
\item[$d=2:$] $1117$.
\item[$d=3:$] $1085$.
\end{description}

\appendix

\chapter{The generator program}

\verbatiminput{copenhagenGraph.hs}

\chapter{CPLEX programs}

\section{Max-edge-independent-paths}

\verbatiminput{maxpaths.lp}

\section{D-max-interest-paths ($d=2$)}

\verbatiminput{algo.lp}

\end{document}
